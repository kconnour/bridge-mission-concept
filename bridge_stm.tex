% This is mediocre Latex code to create the science traceability matrix for the BRIDGE mission designed at JPL, summer 2020

\documentclass[10pt]{book}
\usepackage{cals}
\usepackage[paperwidth=11in, paperheight=9.5in, margin=0.1in]{geometry}
\usepackage{graphicx}

\let\nc=\nullcell                                                  % Shortcuts
\let\sc=\spancontent

% Helpful links
% https://tex.stackexchange.com/questions/66596/vertical-alignment-in-multirow-using-cells-with-1-lines
% https://tex.stackexchange.com/questions/319734/problems-merging-cells
% http://ctan.math.washington.edu/tex-archive/macros/latex/contrib/cals/examples/demo.pdf


\begin{document}

\begin{calstable}
\tiny
% Defining 3 column relativ to each other and relativ to the margins
\colwidths{{1.35in}{1.35in}{1.35in}{1.35in}{1.35in}{1.35in}{1.35in}{1.35in}}

% Set up the tabular
\makeatletter
\newcommand*{\calsNoVertRules}{%
    \renewcommand*{\cals@cs@width}{0pt}%
}

\newcommand*{\calsTopLines}{%
    \renewcommand*{\cals@borderT}{0.4pt}%
}

\newcommand*{\calsBottomLines}{%
    \renewcommand*{\cals@borderB}{0.4pt}%
}

\makeatother
\def\cals@framers@width{0.4pt}   % Outside frame rules, reduce if the rule is too heavy
%\def\cals@framecs@width{0.4pt}
%\def\cals@bodyrs@width{0.4pt}
%\cals@setpadding{Ag}
%\cals@setcellprevdepth{Al}
%\def\cals@cs@width{0.4pt}        % Inside rules, reduce if the rule is too heavy
%\def\cals@rs@width{0.4pt}
%\def\cals@bgcolor{}

%\def\tb{\ifx\cals@borderT\relax     % Top border switch (off-on)
%    \def\cals@borderT{0pt}
%\else \let\cals@borderT\relax\fi}

%\def\bb{\ifx\cals@borderB\relax     % Botton border switch (off-on)
%    \def\cals@borderB{0pt}
%\else \let\cals@borderB\relax\fi}

%\def\rb{\ifx\cals@borderR\relax     % Right border switch (off-on)
%    \def\cals@borderR{0pt}
%\else \let\cals@borderR\relax\fi}

%\def\lb{\ifx\cals@borderL\relax     % Left border switch (off-on)
%    \def\cals@borderL{0pt}
%\else \let\cals@borderL\relax\fi}

\calsNoVertRules
\calsTopLines
\calsBottomLines

%%%% Things learned
% within \nc (\nullcell) you need to include the border (left, top, right, bottom)
% \vfil will center the text vertically
% Text spanning multi rows needs to go at the end of the null cell

% Header
%\brow
%    \nc{ltb}
%    \nc{rtb}
%    \alignC\sc{\vfil Target}
%    \nc{ltb}
%    \nc{rtb}
%    \alignC\sc{\vfil Science Measurement Requirements}
%    \nc{ltb}
%    \nc{tb}
%    \nc{rtb}
%    \alignC\sc{\vfil Instrument Requirements}
%    \nc{rltb}
%    \alignC\sc{\vfil Performance}
%\erow

% Sub-header
\brow
\bfseries
    \alignC \cell{\vfil Science Goal}
    \cell{\vfil Science Objective}
    \cell{\vfil Physical Parameters}
    \cell{\vfil Observables}
    \cell{\vfil Instrument}
    \cell{\vfil Instrument Functional Requirements}
    \cell{\vfil Projected Performance}
    \cell{\vfil Mission Requirements}
\mdseries
\erow

%%%%%%%%%%%%%%%%%%%%%%%%%%%
% Goal 1
%%%%%%%%%%%%%%%%%%%%%%%%%%%

% Row 1 
\brow
    \nc{ltr}
    \nc{ltr}
    \nc{ltr}
    \alignC \cell{\vfil 7.4~$\mu$m (aliphatic), 12.5~$\mu$m (unsaturated), 7--14~$\mu$m (PAH), 7.4~$\mu$m (oxygen groups), 6.5~$\mu$m (nitrogen groups), 7.1~$\mu$m (ketones and carbonyl)}
    \nc{ltr}
    \nc{ltr}
    \nc{ltr}
    \nc{ltr}
\erow

% Row 2
\brow
    \nc{lr}
    \nc{lr}
    \nc{lr}
    \alignC \cell{\vfil 5.3~$\mu$m (tholins)}
    \nc{lbr} \alignC \sc{\vfil Mid-IR spectrometer}
    \nc{lbr} \alignC \sc{\vfil 5--14~$\mu$m spectral range; 0.02~$\mu$m spectral resolution; absorptions $>$1\% in reflectance}
    \nc{lbr} \alignC \sc{\vfil 5--15~$\mu$m spectral range; 0.01~$\mu$m spectral resolution; absorptions $>$0.75\% in reflectance}
    \nc{lr}
\erow

% Row 3
\brow
    \nc{lr}
    \nc{lr}
    \nc{lrb} \alignC \sc{\vfil The presence of functional groups of organic matter. PAHs and tholins are particularly interesting for their biological and space weathering implications.}
    \alignC \cell{\vfil 3.2--3.4~$\mu$m (C-H), 1.6 and 2.5~$\mu$m (O-H), 2.2~$\mu$m ($\mathrm{N_2}$)}
    \nc{ltr}
    \nc{ltr}
    \nc{ltr}
    \nc{lr}
\erow

% Row 4
\brow
    \nc{lr}
    \nc{lr}
    \alignC \cell{\vfil The presence of OH, CH, and $\mathrm{N_2}$ to better than 1\% precision.}
    \alignC \cell{\vfil 2.7--2.8~$\mu$m (O-H)}
    \nc{lbr} \alignC\sc{\vfil Near-IR spectrometer}
    \nc{lbr} \alignC\sc{\vfil 1.6--3.2~$\mu$m spectral range; 0.02~$\mu$m spectral resolution; absorptions $>$1\% in reflectance}
    \nc{lbr} \alignC\sc{\vfil 1--4~$\mu$m spectral range; 0.01~$\mu$m spectral resolution; absorptions $>$0.75\% in reflectance}
    \nc{lbr} \alignC\sc{\vfil 0.3~mrad control (to keep object in field of view). Launch within six months of detection of the ISO.}
\erow

% Row 5
\brow
    \nc{lbr} \alignC\sc{\vfil Determine whether prebiotic chemical ingredients can be dispersed between stellar systems and the interstellar medium.}
    \nc{lbr} \alignC\sc{\vfil Determine whether the ISO contains prebiotic chemical ingredients.}
    \alignC \cell{\vfil The abundance of N, P, and S relative to O.}
    \alignC \cell{\vfil 400~nm (N), 254~nm (P), 420~nm (S), 278~nm (O)}
    \alignC \cell{\vfil UV-visible spectrometer}
    \alignC \cell{\vfil 250--425~nm spectral range; 1~nm spectral resolution; emission $>$1\% continuum}
    \alignC \cell{\vfil 200--600 nm spectral range; 0.4~nm spectral resolution; emission $>$0.1\% continuum}
    \alignC \cell{\vfil Observe at the beginning of the impact through the first 5 seconds of decay. Impact that can be observed by the remote sensing suite with a minimum energy of Deep Impact.}
\erow

%%%%%%%%%%%%%%%%%%%%%%%%%%%
% Goal 2
%%%%%%%%%%%%%%%%%%%%%%%%%%%

% Row 1
\brow
    \nc{ltr}
    \nc{ltr}
    \alignC \cell{\vfil Measurement of $\mathrm{\Delta^{17}O}$.}
    \alignC \cell{\vfil 312.1~nm ($\mathrm{^{16}}$O), 147.7~nm ($\mathrm{^{17}}$O), 312.1~nm ($\mathrm{^{18}}$O)}
    \nc{ltr}
    \alignC \cell{\vfil 307--317~nm and 146--149~nm spectral ranges; 0.03~nm spectral resolution; emission greater than 0.1\% continuum}
    \alignC \cell{\vfil 305--320~nm and 131--151~nm spectral ranges; 0.009~nm spectral resolution; emission greater than 0.1\% continuum in echelle channel}
    \nc{ltr}
\erow

% Row 2
\brow
    \nc{lr}
    \nc{lr}
    \alignC \cell{\vfil Atomic abundances to better than 5\% precision.}
    \alignC \cell{\vfil 567~nm (Si), 264 and 282~nm (Al), 375 and 238~nm (Fe), 393 and 397~nm (Ca), 314 and 589~nm (Na), 208~nm (K), 518~nm (Mg), 336 and 521~nm (Ti), 403~nm (Mn), 299~nm (Ni)}
    \nc{lr}
    \alignC \cell{\vfil 200--590~nm spectral range; 0.5~nm spectral resolution; emission greater than 1\% continuum}
    \nc{ltr}
    \nc{lr}
\erow

% Row 3
\brow
    \nc{lr}
    \nc{lbr} \alignC\sc{\vfil Determine whether the ISO formed in an environment with chemical composition similar to our solar system.}
    \alignC \cell{\vfil Relative abundances of noble gases.}
    \alignC \cell{\vfil 588~nm (He), 540 and 585~nm (Ne), 459 and 473~nm (Ar), 557~nm (Kr), and 481 and 492~nm (Xe)}
    \nc{lbr} \alignC\sc{\vfil UV-visible spectometer}
    \alignC \cell{\vfil 450--590~nm spectral range; 0.5~nm spectral resolution; emission greater than 1\% continuum}
    \nc{lbr} \alignC\sc{\vfil 200--600~nm spectral range; 0.4~nm spectral resolution; emission greater than 0.1\% continuum}
    \nc{lbr} \alignC\sc{\vfil Observe at the beginning of the impact through the first 5 seconds of decay. Impact that can be observed by the remote sensing suite with a minimum energy of Deep Impact.}
\erow

%%%%%%%%%%%%% Objective 3

% Row 4
\brow
    \nc{lr}
    \nc{ltr}
    \nc{ltr}
    \alignC \cell{\vfil 1.6, 1.4, 2.0, and 2.7~$\mu$m ($\mathrm{CO_2}$ ice); 1.05, 1.3, 1.55, 1.65, 2.0, and 3.1~$\mu$m ($\mathrm{H_2O}$ ice); 2.0, 2.1, and 3.0~$\mu$m ($\mathrm{NH_3}$ ice); 1.0, 1.65, 1.82, and 2.2~$\mu$m ($\mathrm{CH_4}$ ice); 2.15~$\mu$m ($\mathrm{N_2}$ ice)}
    \nc{ltr}
    \alignC \cell{\vfil 1.0--3.2~$\mu$m spectral range; 0.02~$\mu$m spectral resolution; absorptions greater than 1\% in reflectance.}
    \nc{ltr}
    \nc{ltr}
\erow

% Row 5
\brow
    \nc{lr}
    \nc{lr}
    \nc{lr}
    \alignC \cell{\vfil 1.0~$\mu$m (olivine), 2.0~$\mu$m (pyroxene), 1.0--1.5~$\mu$m (plagioclase)}
    \nc{lbr} \alignC\sc{\vfil Near-IR specrometer}
    \alignC \cell{\vfil  1.0--2.2~$\mu$m spectral range; 0.02~$\mu$m spectral resolution; absorptions greater than 1\% in reflectance}
    \nc{lbr} \alignC\sc{\vfil 1--4~$\mu$m spectral range; 0.01~$\mu$m spectral resolution; absorptions greater than  0.75\%}
    \nc{lbr} \alignC\sc{\vfil 0.3~mrad control (to keep object in field of view). Launch within six months of detection of the ISO.}
\erow

% Row 6
\brow
    \nc{lr}
    \nc{lr}
    \nc{lbr} \alignC\sc{\vfil Spectral identification of rocks and ices.}
    \alignC \cell{\vfil See above relating to the physical parameter ``Atomic abundances to better than 5\% precision''.}
    \alignC \cell{\vfil UV-visible spectrometer.}
    \alignC \cell{\vfil See above relating to the physical parameter ``Atomic abundances to better than 5\% precision''}
    \alignC \cell{\vfil See above relating to the physical parameter ``Atomic abundances to better than 5\% precision''}
    \nc{ltr}
\erow

% Row 7
\brow
    \nc{lr}
    \nc{lr}
    \nc{ltr}
    \alignC \cell{\vfil 9.7--10.6~$\mu$m (plagioclase)}
    \nc{ltr}
    \alignC \cell{\vfil 9.5--10.8~$\mu$m spectral range; 0.05~$\mu$m spectral resolution; absorptions greater than 1\% in reflectance}
    \nc{ltr}
    \nc{lr}
\erow

% Row 8
\brow
    \nc{lr}
    \nc{lr}
    \nc{lr}
    \alignC \cell{\vfil 9.0--9.8 and 10--12~$\mu$m (clinopyroxene)}
    \nc{lr}
    \alignC \cell{\vfil 8.3--10.5~$\mu$m spectral range; 0.02~$\mu$m spectral resolution; absorptions greater than 1\% in reflectance}
    \nc{lr}
    \nc{lr}
\erow

% Row 9
\brow
    \nc{lr}
    \nc{lr}
    \nc{lr}
    \alignC \cell{\vfil 9.5--12.0~$\mu$m (olivine)}
    \nc{lr}
    \alignC \cell{\vfil 9.0--12.0~$\mu$m spectral range; 0.1~$\mu$m spectral resolution; absorptions greater than 1\% in reflectance}
    \nc{lr}
    \nc{lr}
\erow

% Row 10
\brow
    \nc{lr}
    \nc{lr}
    \nc{lr}
    \alignC \cell{\vfil 13.1~$\mu$m (oxides)}
    \nc{lbr} \alignC\sc{\vfil Mid-IR spectrometer}
    \alignC \cell{\vfil 12.0--14.0~$\mu$m spectral range; 0.02~$\mu$m spectral resolution; absorptions greater than 1\% in reflectance}
    \nc{lbr} \alignC\sc{\vfil 5--15~$\mu$m spectral range; 0.01~$\mu$m spectral resolution; absorptions greater than 0.75\%}
    \nc{lr}
\erow

% Row 11
\brow
    \nc{lr}
    \nc{lr}
    \nc{lbr} \alignC\sc{\vfil Molar abundances of minerals to the 1\% precision level.}
    \alignC \cell{\vfil See above relating to the physical parameter ``The abundance of N, P, and S relative to O.''}
    \alignC \cell{\vfil UV-visible spectrometer}
    \alignC \cell{\vfil See above relating to the physical parameter ``The abundance of N, P, and S relative to O.''}
    \alignC \cell{\vfil See above relating to the physical parameter ``The abundance of N, P, and S relative to O.''}
    \nc{lbr} \alignC\sc{\vfil Observe at the beginning of the impact through the first 5 seconds of decay. Impact that can be observed by the remote sensing suite with a minimum energy of Deep Impact.}
\erow

% Row 12
\brow
    \nc{lr}
    \nc{lr}
    \alignC \cell{\vfil The ratio of rock/ice in the surface exposure to the 1\% precision level.}
    \alignC \cell{\vfil Spatial variation in 350--850~nm intensity.}
    \alignC \cell{\vfil Impactor camera}
    \alignC \cell{\vfil 10~m/pixel at closest approach}
    \alignC \cell{\vfil 7.5~m/pixel at closest approach}
    \alignC \cell{\vfil Observe at the time leading up to impact.}
\erow

% Row 13
\brow
    \nc{lrb} \alignC\sc{\vfil Determine whether interstellar objects form via the same processes that created objects within our solar system.}
    \nc{lrb} \alignC\sc{\vfil Determine if processes in extrasolar systems lead to bodies similar to those found in our solar system.}
    \alignC \cell{\vfil Bulk morphological properties of the ISO to 10~m resolution.}
    \alignC \cell{\vfil Intensity and its variation between 300--700~nm.}
    \alignC \cell{\vfil Spacecraft camera}
    \alignC \cell{\vfil 25~m/pixel at closest approach}
    \alignC \cell{\vfil 20~m/pixel at closest approach}
    \alignC \cell{\vfil 0.3~mrad control (to keep object in field of view). Launch within six months of detection of the ISO. Camera slew of 0.43~deg/sec at closest approach.}
\erow

\end{calstable}\par

\end{document}
